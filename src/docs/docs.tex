\documentclass{article}[11pt]
\usepackage[subtle]{savetrees}
\usepackage[left=1in, right=1in, top=1in, bottom=1in]{geometry}

\usepackage{amsthm}
\usepackage{amssymb}
\usepackage{amsmath}
\usepackage{mathtools}

\usepackage{fancyhdr}
\pagestyle{fancy}
\lhead{Alek Westover}
\rhead{}
\usepackage{hyperref}

\DeclareMathOperator{\E}{\mathbb{E}}
\DeclareMathOperator{\Var}{\text{Var}}
\DeclareMathOperator{\img}{Im}
\DeclareMathOperator{\polylog}{\text{polylog}}
\DeclareMathOperator{\st}{\text{ such that }}
\newcommand{\norm}[1]{\left\lVert#1\right\rVert}
\newcommand{\interior}[1]{%
  {\kern0pt#1}^{\mathrm{o}}%
}

%** VECTOR NOTATION
\newcommand{\mb}{\mathbf}
\newcommand{\x}{\mathbf{x}}
\newcommand{\y}{\mathbf{y}}
\newcommand{\z}{\mathbf{z}}
\newcommand{\f}{\mathbf{f}}
\newcommand{\n}{\mathbf{n}}
\newcommand{\p}{\mathbf{p}}
\renewcommand{\k}{\mathbf{k}}
\renewcommand{\d}{\mathrm{d}} %straight d for integrals
\newcommand{\De}{\Delta}
\renewcommand{\Re}{\mathrm{Re}}
\renewcommand{\Im}{\mathrm{Im}}
\newcommand{\ran}{\mathrm{ran}}

%** SETS
\newcommand{\set}[1]{\mathbb{#1}}
\newcommand{\curly}[1]{\mathcal{#1}}
\newcommand{\goth}[1]{\mathfrak{#1}}
\newcommand{\setof}[2]{\left\{ #1\; : \;#2 \right\}}
\newcommand{\cc}{\subseteq\subseteq}
\newcommand{\R}{\set{R}}
\newcommand{\C}{\set{C}}
\newcommand{\Z}{\set{Z}}
\newcommand{\D}{\curly{D}}
\renewcommand{\S}{\set{S}}
\newcommand{\T}{\set{T}}


\newcommand{\contr}[0]{\[ \Rightarrow\!\Leftarrow \]}
\newcommand{\defeq}{\vcentcolon=}
\newcommand{\eqdef}{=\vcentcolon}

\newtheorem{fact}{Fact}
\newtheorem{definition}{Definition}
\newtheorem{remark}{Remark}
\newtheorem{proposition}{Proposition}
\newtheorem{lemma}{Lemma}
\newtheorem{corollary}{Corollary}
\newtheorem{theorem}{Theorem}

\usepackage{mdframed}
\newmdtheoremenv{q}{Question}

\DeclareMathOperator{\ord}{\textbf{ord}}
\DeclareMathOperator{\chr}{\textbf{chr}}
\DeclareMathOperator{\str}{\textbf{str}}
\DeclareMathOperator{\into}{\textbf{int}}
\DeclareMathOperator{\divmod}{\textbf{divmod}}

\usepackage{tikz-cd}

\begin{document}
\begin{center}
	\textbf{Documentation:}
\end{center}

There are many different methods that we use to specify a location on the
board.  They are all often convenient for different reasons. The three ways of
representing a location on the board are:

\begin{itemize}
	\item \emph{human	readable:} e.g. a1. This is the normal chess lingo. It is used for I/O.
	\item \emph{2d grid}: e.g. (1,1). This is useful for specifying moves.
	\item \emph{flattened grid}: e.g. 4. The board is stored as a string. this is kind of
		nice, but maybe was a bad idea. Think of it as an imutable 1d array though
		(because that's what it is).
\end{itemize} 

\vspace{1cm}
The transformations between states are depicted in the following diagram:
\vspace{0.25cm}
$$
\begin{tikzcd}
	\text{Human Readable} \quad \arrow[r, yshift=-3ex, "update"] 
	&\quad \text{2d grid}\quad \arrow[l, yshift=3ex, "humanify"] \arrow[r, yshift=-3ex, "ij\_to\_idx"] 
	&\quad \text{1d grid} \arrow[l, yshift=3ex, "idx\_to\_ij"]
\end{tikzcd}
$$
\vspace{1cm}
$ $\\
Here is an outline of the important functions:\\
\textbf{humanifty:}
$$(i,j) \quad\mapsto\quad \chr(\ord(\text{`a'})+j) + \str(4-i)$$
\textbf{update:}
(note that this function is ``in-place'')
$$h_0h_1 \quad \mapsto \quad (4-\into(h_1), \ord(h_0) - \ord(\text{`a'}))$$
\textbf{ij\_to\_idx:}
$$(i,j) \quad \mapsto \quad 4i+j$$
\textbf{idx\_to\_ij:}
$$\text{idx} \quad \mapsto \quad \divmod(\text{idx})$$

\vspace{1cm}
$ $\\
Here is what the grid looks like:
\begin{table}[h]
\centering
\begin{tabular}{lllll}
\cline{2-5}
\multicolumn{1}{l|}{4} & \multicolumn{1}{l|}{(0,0)=0}  & \multicolumn{1}{l|}{(0,1)=1}  & \multicolumn{1}{l|}{(0,2)=2}  & \multicolumn{1}{l|}{(0,3)=3}  \\ \cline{2-5} 
\multicolumn{1}{l|}{3} & \multicolumn{1}{l|}{(1,0)=4}  & \multicolumn{1}{l|}{(1,1)=5}  & \multicolumn{1}{l|}{(1,2)=6}  & \multicolumn{1}{l|}{(1,3)=7}  \\ \cline{2-5} 
\multicolumn{1}{l|}{2} & \multicolumn{1}{l|}{(2,0)=8}  & \multicolumn{1}{l|}{(2,1)=9}  & \multicolumn{1}{l|}{(2,2)=10} & \multicolumn{1}{l|}{(2,3)=11} \\ \cline{2-5} 
\multicolumn{1}{l|}{1} & \multicolumn{1}{l|}{(3,0)=12} & \multicolumn{1}{l|}{(3,1)=13} & \multicolumn{1}{l|}{(3,2)=14} & \multicolumn{1}{l|}{(3,3)=15} \\ \cline{2-5} 
                       & a                             & b                             & c                             & d                            
\end{tabular}
\end{table}
% https://www.tablesgenerator.com/latex_tables#


\end{document}

